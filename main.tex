
%% bare_adv.tex
%% V1.4
%% 2012/12/27
%% by Michael Shell
%% See: 
%% http://www.michaelshell.org/
%% for current contact information.
%%
%% This is a skeleton file demonstrating the advanced use of IEEEtran.cls
%% (requires IEEEtran.cls version 1.8 or later) with an IEEE Computer
%% Society journal paper.
%%
%% Support sites:
%% http://www.michaelshell.org/tex/ieeetran/
%% http://www.ctan.org/tex-archive/macros/latex/contrib/IEEEtran/
%% and
%% http://www.ieee.org/

%%*************************************************************************
%% Legal Notice:
%% This code is offered as-is without any warranty either expressed or
%% implied; without even the implied warranty of MERCHANTABILITY or
%% FITNESS FOR A PARTICULAR PURPOSE! 
%% User assumes all risk.
%% In no event shall IEEE or any contributor to this code be liable for
%% any damages or losses, including, but not limited to, incidental,
%% consequential, or any other damages, resulting from the use or misuse
%% of any information contained here.
%%
%% All comments are the opinions of their respective authors and are not
%% necessarily endorsed by the IEEE.
%%
%% This work is distributed under the LaTeX Project Public License (LPPL)
%% ( http://www.latex-project.org/ ) version 1.3, and may be freely used,
%% distributed and modified. A copy of the LPPL, version 1.3, is included
%% in the base LaTeX documentation of all distributions of LaTeX released
%% 2003/12/01 or later.
%% Retain all contribution notices and credits.
%% ** Modified files should be clearly indicated as such, including  **
%% ** renaming them and changing author support contact information. **
%%
%% File list of work: IEEEtran.cls, IEEEtran_HOWTO.pdf, bare_adv.tex,
%%                    bare_conf.tex, bare_jrnl.tex, bare_jrnl_compsoc.tex,
%%                    bare_jrnl_transmag.tex
%%*************************************************************************

% *** Authors should verify (and, if needed, correct) their LaTeX system  ***
% *** with the testflow diagnostic prior to trusting their LaTeX platform ***
% *** with production work. IEEE's font choices can trigger bugs that do  ***
% *** not appear when using other class files.                            ***
% The testflow support page is at:
% http://www.michaelshell.org/tex/testflow/



% IEEEtran V1.7 and later provides for these CLASSINPUT macros to allow the
% user to reprogram some IEEEtran.cls defaults if needed. These settings
% override the internal defaults of IEEEtran.cls regardless of which class
% options are used. Do not use these unless you have good reason to do so as
% they can result in nonIEEE compliant documents. User beware. ;)
%
%\newcommand{\CLASSINPUTbaselinestretch}{1.0} % baselinestretch
%\newcommand{\CLASSINPUTinnersidemargin}{1in} % inner side margin
%\newcommand{\CLASSINPUToutersidemargin}{1in} % outer side margin
%\newcommand{\CLASSINPUTtoptextmargin}{1in}   % top text margin
%\newcommand{\CLASSINPUTbottomtextmargin}{1in}% bottom text margin



% Note that the a4paper option is mainly intended so that authors in
% countries using A4 can easily print to A4 and see how their papers will
% look in print - the typesetting of the document will not typically be
% affected with changes in paper size (but the bottom and side margins will).
% Use the testflow package mentioned above to verify correct handling of
% both paper sizes by the user's LaTeX system.
%
% Also note that the "draftcls" or "draftclsnofoot", not "draft", option
% should be used if it is desired that the figures are to be displayed in
% draft mode.
%
\documentclass[12pt,journal,compsoc]{IEEEtran}
% The Computer Society requires 12pt.
% If IEEEtran.cls has not been installed into the LaTeX system files,
% manually specify the path to it like:
% \documentclass[10pt,journal,compsoc]{../sty/IEEEtran}


% For Computer Society journals, IEEEtran defaults to the use of 
% Palatino/Palladio as is done in IEEE Computer Society journals.
% To go back to Times Roman, you can use this code:
%\renewcommand{\rmdefault}{ptm}\selectfont





% Some very useful LaTeX packages include:
% (uncomment the ones you want to load)



% *** MISC UTILITY PACKAGES ***
%
%\usepackage{ifpdf}
% Heiko Oberdiek's ifpdf.sty is very useful if you need conditional
% compilation based on whether the output is pdf or dvi.
% usage:
% \ifpdf
%   % pdf code
% \else
%   % dvi code
% \fi
% The latest version of ifpdf.sty can be obtained from:
% http://www.ctan.org/tex-archive/macros/latex/contrib/oberdiek/
% Also, note that IEEEtran.cls V1.7 and later provides a builtin
% \ifCLASSINFOpdf conditional that works the same way.
% When switching from latex to pdflatex and vice-versa, the compiler may
% have to be run twice to clear warning/error messages.






% *** CITATION PACKAGES ***
%
\ifCLASSOPTIONcompsoc
  % IEEE Computer Society needs nocompress option
  % requires cite.sty v4.0 or later (November 2003)
  % \usepackage[nocompress]{cite}
\else
  % normal IEEE
  % \usepackage{cite}
\fi
% cite.sty was written by Donald Arseneau
% V1.6 and later of IEEEtran pre-defines the format of the cite.sty package
% \cite{} output to follow that of IEEE. Loading the cite package will
% result in citation numbers being automatically sorted and properly
% "compressed/ranged". e.g., [1], [9], [2], [7], [5], [6] without using
% cite.sty will become [1], [2], [5]--[7], [9] using cite.sty. cite.sty's
% \cite will automatically add leading space, if needed. Use cite.sty's
% noadjust option (cite.sty V3.8 and later) if you want to turn this off
% such as if a citation ever needs to be enclosed in parenthesis.
% cite.sty is already installed on most LaTeX systems. Be sure and use
% version 4.0 (2003-05-27) and later if using hyperref.sty. cite.sty does
% not currently provide for hyperlinked citations.
% The latest version can be obtained at:
% http://www.ctan.org/tex-archive/macros/latex/contrib/cite/
% The documentation is contained in the cite.sty file itself.
%
% Note that some packages require special options to format as the Computer
% Society requires. In particular, Computer Society  papers do not use
% compressed citation ranges as is done in typical IEEE papers
% (e.g., [1]-[4]). Instead, they list every citation separately in order
% (e.g., [1], [2], [3], [4]). To get the latter we need to load the cite
% package with the nocompress option which is supported by cite.sty v4.0
% and later.
%
% Note also the use of a CLASSOPTION conditional provided by 
% IEEEtran.cls V1.7 and later.





% *** GRAPHICS RELATED PACKAGES ***
%
\ifCLASSINFOpdf
  % \usepackage[pdftex]{graphicx}
  % declare the path(s) where your graphic files are
  % \graphicspath{{../pdf/}{../jpeg/}}
  % and their extensions so you won't have to specify these with
  % every instance of \includegraphics
  % \DeclareGraphicsExtensions{.pdf,.jpeg,.png}
\else
  % or other class option (dvipsone, dvipdf, if not using dvips). graphicx
  % will default to the driver specified in the system graphics.cfg if no
  % driver is specified.
  % \usepackage[dvips]{graphicx}
  % declare the path(s) where your graphic files are
  % \graphicspath{{../eps/}}
  % and their extensions so you won't have to specify these with
  % every instance of \includegraphics
  % \DeclareGraphicsExtensions{.eps}
\fi
% graphicx was written by David Carlisle and Sebastian Rahtz. It is
% required if you want graphics, photos, etc. graphicx.sty is already
% installed on most LaTeX systems. The latest version and documentation
% can be obtained at: 
% http://www.ctan.org/tex-archive/macros/latex/required/graphics/
% Another good source of documentation is "Using Imported Graphics in
% LaTeX2e" by Keith Reckdahl which can be found at:
% http://www.ctan.org/tex-archive/info/epslatex/
%
% latex, and pdflatex in dvi mode, support graphics in encapsulated
% postscript (.eps) format. pdflatex in pdf mode supports graphics
% in .pdf, .jpeg, .png and .mps (metapost) formats. Users should ensure
% that all non-photo figures use a vector format (.eps, .pdf, .mps) and
% not a bitmapped formats (.jpeg, .png). IEEE frowns on bitmapped formats
% which can result in "jaggedy"/blurry rendering of lines and letters as
% well as large increases in file sizes.
%
% You can find documentation about the pdfTeX application at:
% http://www.tug.org/applications/pdftex





% *** MATH PACKAGES ***
%
%\usepackage[cmex10]{amsmath}
% A popular package from the American Mathematical Society that provides
% many useful and powerful commands for dealing with mathematics. If using
% it, be sure to load this package with the cmex10 option to ensure that
% only type 1 fonts will utilized at all point sizes. Without this option,
% it is possible that some math symbols, particularly those within
% footnotes, will be rendered in bitmap form which will result in a
% document that can not be IEEE Xplore compliant!
%
% Also, note that the amsmath package sets \interdisplaylinepenalty to 10000
% thus preventing page breaks from occurring within multiline equations. Use:
%\interdisplaylinepenalty=2500
% after loading amsmath to restore such page breaks as IEEEtran.cls normally
% does. amsmath.sty is already installed on most LaTeX systems. The latest
% version and documentation can be obtained at:
% http://www.ctan.org/tex-archive/macros/latex/required/amslatex/math/





% *** SPECIALIZED LIST PACKAGES ***
%\usepackage{acronym}
% acronym.sty was written by Tobias Oetiker. This package provides tools for
% managing documents with large numbers of acronyms. (You don't *have* to
% use this package - unless you have a lot of acronyms, you may feel that
% such package management of them is bit of an overkill.)
% Do note that the acronym environment (which lists acronyms) will have a
% problem when used under IEEEtran.cls because acronym.sty relies on the
% description list environment - which IEEEtran.cls has customized for
% producing IEEE style lists. A workaround is to declared the longest
% label width via the IEEEtran.cls \IEEEiedlistdecl global control:
%
% \renewcommand{\IEEEiedlistdecl}{\IEEEsetlabelwidth{SONET}}
% \begin{acronym}
%
% \end{acronym}
% \renewcommand{\IEEEiedlistdecl}{\relax}% remember to reset \IEEEiedlistdecl
%
% instead of using the acronym environment's optional argument.
% The latest version and documentation can be obtained at:
% http://www.ctan.org/tex-archive/macros/latex/contrib/acronym/


%\usepackage{algorithmic}
% algorithmic.sty was written by Peter Williams and Rogerio Brito.
% This package provides an algorithmic environment fo describing algorithms.
% You can use the algorithmic environment in-text or within a figure
% environment to provide for a floating algorithm. Do NOT use the algorithm
% floating environment provided by algorithm.sty (by the same authors) or
% algorithm2e.sty (by Christophe Fiorio) as IEEE does not use dedicated
% algorithm float types and packages that provide these will not provide
% correct IEEE style captions. The latest version and documentation of
% algorithmic.sty can be obtained at:
% http://www.ctan.org/tex-archive/macros/latex/contrib/algorithms/
% There is also a support site at:
% http://algorithms.berlios.de/index.html
% Also of interest may be the (relatively newer and more customizable)
% algorithmicx.sty package by Szasz Janos:
% http://www.ctan.org/tex-archive/macros/latex/contrib/algorithmicx/




% *** ALIGNMENT PACKAGES ***
%
%\usepackage{array}
% Frank Mittelbach's and David Carlisle's array.sty patches and improves
% the standard LaTeX2e array and tabular environments to provide better
% appearance and additional user controls. As the default LaTeX2e table
% generation code is lacking to the point of almost being broken with
% respect to the quality of the end results, all users are strongly
% advised to use an enhanced (at the very least that provided by array.sty)
% set of table tools. array.sty is already installed on most systems. The
% latest version and documentation can be obtained at:
% http://www.ctan.org/tex-archive/macros/latex/required/tools/


%\usepackage{mdwmath}
%\usepackage{mdwtab}
% Also highly recommended is Mark Wooding's extremely powerful MDW tools,
% especially mdwmath.sty and mdwtab.sty which are used to format equations
% and tables, respectively. The MDWtools set is already installed on most
% LaTeX systems. The lastest version and documentation is available at:
% http://www.ctan.org/tex-archive/macros/latex/contrib/mdwtools/


% IEEEtran contains the IEEEeqnarray family of commands that can be used to
% generate multiline equations as well as matrices, tables, etc., of high
% quality.


%\usepackage{eqparbox}
% Also of notable interest is Scott Pakin's eqparbox package for creating
% (automatically sized) equal width boxes - aka "natural width parboxes".
% Available at:
% http://www.ctan.org/tex-archive/macros/latex/contrib/eqparbox/




% *** SUBFIGURE PACKAGES ***
%\ifCLASSOPTIONcompsoc
%  \usepackage[caption=false,font=normalsize,labelfont=sf,textfont=sf]{subfig}
%\else
%  \usepackage[caption=false,font=footnotesize]{subfig}
%\fi
% subfig.sty, written by Steven Douglas Cochran, is the modern replacement
% for subfigure.sty, the latter of which is no longer maintained and is
% incompatible with some LaTeX packages including fixltx2e. However,
% subfig.sty requires and automatically loads Axel Sommerfeldt's caption.sty
% which will override IEEEtran.cls' handling of captions and this will result
% in non-IEEE style figure/table captions. To prevent this problem, be sure
% and invoke subfig.sty's "caption=false" package option (available since
% subfig.sty version 1.3, 2005/06/28) as this is will preserve IEEEtran.cls
% handling of captions.
% Note that the Computer Society format requires a larger sans serif font
% than the serif footnote size font used in traditional IEEE formatting
% and thus the need to invoke different subfig.sty package options depending
% on whether compsoc mode has been enabled.
%
% The latest version and documentation of subfig.sty can be obtained at:
% http://www.ctan.org/tex-archive/macros/latex/contrib/subfig/




% *** FLOAT PACKAGES ***
%
%\usepackage{fixltx2e}
% fixltx2e, the successor to the earlier fix2col.sty, was written by
% Frank Mittelbach and David Carlisle. This package corrects a few problems
% in the LaTeX2e kernel, the most notable of which is that in current
% LaTeX2e releases, the ordering of single and double column floats is not
% guaranteed to be preserved. Thus, an unpatched LaTeX2e can allow a
% single column figure to be placed prior to an earlier double column
% figure. The latest version and documentation can be found at:
% http://www.ctan.org/tex-archive/macros/latex/base/


%\usepackage{stfloats}
% stfloats.sty was written by Sigitas Tolusis. This package gives LaTeX2e
% the ability to do double column floats at the bottom of the page as well
% as the top. (e.g., "\begin{figure*}[!b]" is not normally possible in
% LaTeX2e). It also provides a command:
%\fnbelowfloat
% to enable the placement of footnotes below bottom floats (the standard
% LaTeX2e kernel puts them above bottom floats). This is an invasive package
% which rewrites many portions of the LaTeX2e float routines. It may not work
% with other packages that modify the LaTeX2e float routines. The latest
% version and documentation can be obtained at:
% http://www.ctan.org/tex-archive/macros/latex/contrib/sttools/
% Do not use the stfloats baselinefloat ability as IEEE does not allow
% \baselineskip to stretch. Authors submitting work to the IEEE should note
% that IEEE rarely uses double column equations and that authors should try
% to avoid such use. Do not be tempted to use the cuted.sty or midfloat.sty
% packages (also by Sigitas Tolusis) as IEEE does not format its papers in
% such ways.
% Do not attempt to use stfloats with fixltx2e as they are incompatible.
% Instead, use Morten Hogholm'a dblfloatfix which combines the features
% of both fixltx2e and stfloats:
%
% \usepackage{dblfloatfix}
% The latest version can be found at:
% http://www.ctan.org/tex-archive/macros/latex/contrib/dblfloatfix/


%\ifCLASSOPTIONcaptionsoff
%  \usepackage[nomarkers]{endfloat}
% \let\MYoriglatexcaption\caption
% \renewcommand{\caption}[2][\relax]{\MYoriglatexcaption[#2]{#2}}
%\fi
% endfloat.sty was written by James Darrell McCauley, Jeff Goldberg and 
% Axel Sommerfeldt. This package may be useful when used in conjunction with 
% IEEEtran.cls'  captionsoff option. Some IEEE journals/societies require that
% submissions have lists of figures/tables at the end of the paper and that
% figures/tables without any captions are placed on a page by themselves at
% the end of the document. If needed, the draftcls IEEEtran class option or
% \CLASSINPUTbaselinestretch interface can be used to increase the line
% spacing as well. Be sure and use the nomarkers option of endfloat to
% prevent endfloat from "marking" where the figures would have been placed
% in the text. The two hack lines of code above are a slight modification of
% that suggested by in the endfloat docs (section 8.4.1) to ensure that
% the full captions always appear in the list of figures/tables - even if
% the user used the short optional argument of \caption[]{}.
% IEEE papers do not typically make use of \caption[]'s optional argument,
% so this should not be an issue. A similar trick can be used to disable
% captions of packages such as subfig.sty that lack options to turn off
% the subcaptions:
% For subfig.sty:
% \let\MYorigsubfloat\subfloat
% \renewcommand{\subfloat}[2][\relax]{\MYorigsubfloat[]{#2}}
% However, the above trick will not work if both optional arguments of
% the \subfloat command are used. Furthermore, there needs to be a
% description of each subfigure *somewhere* and endfloat does not add
% subfigure captions to its list of figures. Thus, the best approach is to
% avoid the use of subfigure captions (many IEEE journals avoid them anyway)
% and instead reference/explain all the subfigures within the main caption.
% The latest version of endfloat.sty and its documentation can obtained at:
% http://www.ctan.org/tex-archive/macros/latex/contrib/endfloat/
%
% The IEEEtran \ifCLASSOPTIONcaptionsoff conditional can also be used
% later in the document, say, to conditionally put the References on a 
% page by themselves.





% *** PDF, URL AND HYPERLINK PACKAGES ***
%
%\usepackage{url}
% url.sty was written by Donald Arseneau. It provides better support for
% handling and breaking URLs. url.sty is already installed on most LaTeX
% systems. The latest version and documentation can be obtained at:
% http://www.ctan.org/tex-archive/macros/latex/contrib/url/
% Basically, \url{my_url_here}.


% NOTE: PDF thumbnail features are not required in IEEE papers
%       and their use requires extra complexity and work.
%\ifCLASSINFOpdf
%  \usepackage[pdftex]{thumbpdf}
%\else
%  \usepackage[dvips]{thumbpdf}
%\fi
% thumbpdf.sty and its companion Perl utility were written by Heiko Oberdiek.
% It allows the user a way to produce PDF documents that contain fancy
% thumbnail images of each of the pages (which tools like acrobat reader can
% utilize). This is possible even when using dvi->ps->pdf workflow if the
% correct thumbpdf driver options are used. thumbpdf.sty incorporates the
% file containing the PDF thumbnail information (filename.tpm is used with
% dvips, filename.tpt is used with pdftex, where filename is the base name of
% your tex document) into the final ps or pdf output document. An external
% utility, the thumbpdf *Perl script* is needed to make these .tpm or .tpt
% thumbnail files from a .ps or .pdf version of the document (which obviously
% does not yet contain pdf thumbnails). Thus, one does a:
% 
% thumbpdf filename.pdf 
%
% to make a filename.tpt, and:
%
% thumbpdf --mode dvips filename.ps
%
% to make a filename.tpm which will then be loaded into the document by
% thumbpdf.sty the NEXT time the document is compiled (by pdflatex or
% latex->dvips->ps2pdf). Users must be careful to regenerate the .tpt and/or
% .tpm files if the main document changes and then to recompile the
% document to incorporate the revised thumbnails to ensure that thumbnails
% match the actual pages. It is easy to forget to do this!
% 
% Unix systems come with a Perl interpreter. However, MS Windows users
% will usually have to install a Perl interpreter so that the thumbpdf
% script can be run. The Ghostscript PS/PDF interpreter is also required.
% See the thumbpdf docs for details. The latest version and documentation
% can be obtained at.
% http://www.ctan.org/tex-archive/support/thumbpdf/


% NOTE: PDF hyperlink and bookmark features are not required in IEEE
%       papers and their use requires extra complexity and work.
% *** IF USING HYPERREF BE SURE AND CHANGE THE EXAMPLE PDF ***
% *** TITLE/SUBJECT/AUTHOR/KEYWORDS INFO BELOW!!           ***
\newcommand\MYhyperrefoptions{bookmarks=true,bookmarksnumbered=true,
pdfpagemode={UseOutlines},plainpages=false,pdfpagelabels=true,
colorlinks=true,linkcolor={black},citecolor={black},urlcolor={black},
pdftitle={Attention Based Stock Price Prediction},%<!CHANGE!
pdfsubject={Journal paper},%<!CHANGE!
pdfauthor={Ahmad Asadi, Ehsan Hajizadeh, Reza Safabakhsh},%<!CHANGE!
pdfkeywords={Computer Society, IEEEtran, journal, IEEE, paper,
             manuscript}}%<^!CHANGE!
%\ifCLASSINFOpdf
%\usepackage[\MYhyperrefoptions,pdftex]{hyperref}
%\else
%\usepackage[\MYhyperrefoptions,breaklinks=true,dvips]{hyperref}
%\usepackage{breakurl}
%\fi
% One significant drawback of using hyperref under DVI output is that the
% LaTeX compiler cannot break URLs across lines or pages as can be done
% under pdfLaTeX's PDF output via the hyperref pdftex driver. This is
% probably the single most important capability distinction between the
% DVI and PDF output. Perhaps surprisingly, all the other PDF features
% (PDF bookmarks, thumbnails, etc.) can be preserved in
% .tex->.dvi->.ps->.pdf workflow if the respective packages/scripts are
% loaded/invoked with the correct driver options (dvips, etc.). 
% As most IEEE papers use URLs sparingly (mainly in the references), this
% may not be as big an issue as with other publications.
%
% That said, Vilar Camara Neto created his breakurl.sty package which
% permits hyperref to easily break URLs even in dvi mode.
% Note that breakurl, unlike most other packages, must be loaded
% AFTER hyperref. The latest version of breakurl and its documentation can
% be obtained at:
% http://www.ctan.org/tex-archive/macros/latex/contrib/breakurl/
% breakurl.sty is not for use under pdflatex pdf mode.
%
% The advanced features offer by hyperref.sty are not required for IEEE
% submission, so users should weigh these features against the added
% complexity of use.
% The package options above demonstrate how to enable PDF bookmarks
% (a type of table of contents viewable in Acrobat Reader) as well as
% PDF document information (title, subject, author and keywords) that is
% viewable in Acrobat reader's Document_Properties menu. PDF document
% information is also used extensively to automate the cataloging of PDF
% documents. The above set of options ensures that hyperlinks will not be
% colored in the text and thus will not be visible in the printed page,
% but will be active on "mouse over". USING COLORS OR OTHER HIGHLIGHTING
% OF HYPERLINKS CAN RESULT IN DOCUMENT REJECTION BY THE IEEE, especially if
% these appear on the "printed" page. IF IN DOUBT, ASK THE RELEVANT
% SUBMISSION EDITOR. You may need to add the option hypertexnames=false if
% you used duplicate equation numbers, etc., but this should not be needed
% in normal IEEE work.
% The latest version of hyperref and its documentation can be obtained at:
% http://www.ctan.org/tex-archive/macros/latex/contrib/hyperref/





% *** Do not adjust lengths that control margins, column widths, etc. ***
% *** Do not use packages that alter fonts (such as pslatex).         ***
% There should be no need to do such things with IEEEtran.cls V1.6 and later.
% (Unless specifically asked to do so by the journal or conference you plan
% to submit to, of course. )


% correct bad hyphenation here
\hyphenation{op-tical net-works semi-conduc-tor}


\begin{document}
%
% paper title
% can use linebreaks \\ within to get better formatting as desired
% Do not put math or special symbols in the title.
\title{Attention Based Stock Price Prediction}
%
%
% author names and IEEE memberships
% note positions of commas and nonbreaking spaces ( ~ ) LaTeX will not break
% a structure at a ~ so this keeps an author's name from being broken across
% two lines.
% use \thanks{} to gain access to the first footnote area
% a separate \thanks must be used for each paragraph as LaTeX2e's \thanks
% was not built to handle multiple paragraphs
%
%
%\IEEEcompsocitemizethanks is a special \thanks that produces the bulleted
% lists the Computer Society journals use for "first footnote" author
% affiliations. Use \IEEEcompsocthanksitem which works much like \item
% for each affiliation group. When not in compsoc mode,
% \IEEEcompsocitemizethanks becomes like \thanks and
% \IEEEcompsocthanksitem becomes a line break with idention. This
% facilitates dual compilation, although admittedly the differences in the
% desired content of \author between the different types of papers makes a
% one-size-fits-all approach a daunting prospect. For instance, compsoc 
% journal papers have the author affiliations above the "Manuscript
% received ..."  text while in non-compsoc journals this is reversed. Sigh.

\author{Ahmad~Asadi,
        Ehsan~Hajizadeh,
        and~Reza~Safabakhsh% <-this % stops a space
\IEEEcompsocitemizethanks{\IEEEcompsocthanksitem M. Shell is with the Department
of Computer Engineering, Amirkabir University of Technology, Iran.\protect\\
% note need leading \protect in front of \\ to get a newline within \thanks as
% \\ is fragile and will error, could use \hfil\break instead.
E-mail: ahmad.asadi@aut.ac.ir
\IEEEcompsocthanksitem J. Doe and J. Doe are with Anonymous University.}% <-this % stops a space
\thanks{Manuscript received April 19, 2005; revised December 27, 2012.}}

% note the % following the last \IEEEmembership and also \thanks - 
% these prevent an unwanted space from occurring between the last author name
% and the end of the author line. i.e., if you had this:
% 
% \author{....lastname \thanks{...} \thanks{...} }
%                     ^------------^------------^----Do not want these spaces!
%
% a space would be appended to the last name and could cause every name on that
% line to be shifted left slightly. This is one of those "LaTeX things". For
% instance, "\textbf{A} \textbf{B}" will typeset as "A B" not "AB". To get
% "AB" then you have to do: "\textbf{A}\textbf{B}"
% \thanks is no different in this regard, so shield the last } of each \thanks
% that ends a line with a % and do not let a space in before the next \thanks.
% Spaces after \IEEEmembership other than the last one are OK (and needed) as
% you are supposed to have spaces between the names. For what it is worth,
% this is a minor point as most people would not even notice if the said evil
% space somehow managed to creep in.



% The paper headers
\markboth{Journal of \LaTeX\ Class Files,~Vol.~11, No.~4, December~2012}%
{Asadi \MakeLowercase{\textit{et al.}}: Attention Based Stock Price Prediction}
% The only time the second header will appear is for the odd numbered pages
% after the title page when using the twoside option.
% 
% *** Note that you probably will NOT want to include the author's ***
% *** name in the headers of peer review papers.                   ***
% You can use \ifCLASSOPTIONpeerreview for conditional compilation here if
% you desire.



% The publisher's ID mark at the bottom of the page is less important with
% Computer Society journal papers as those publications place the marks
% outside of the main text columns and, therefore, unlike regular IEEE
% journals, the available text space is not reduced by their presence.
% If you want to put a publisher's ID mark on the page you can do it like
% this:
%\IEEEpubid{0000--0000/00\$00.00~\copyright~2012 IEEE}
% or like this to get the Computer Society new two part style.
%\IEEEpubid{\makebox[\columnwidth]{\hfill 0000--0000/00/\$00.00~\copyright~2012 IEEE}%
%\hspace{\columnsep}\makebox[\columnwidth]{Published by the IEEE Computer Society\hfill}}
% Remember, if you use this you must call \IEEEpubidadjcol in the second
% column for its text to clear the IEEEpubid mark (Computer Society journal
% papers don't need this extra clearance.)



% use for special paper notices
%\IEEEspecialpapernotice{(Invited Paper)}



% for Computer Society papers, we must declare the abstract and index terms
% PRIOR to the title within the \IEEEtitleabstractindextext IEEEtran
% command as these need to go into the title area created by \maketitle.
% As a general rule, do not put math, special symbols or citations
% in the abstract or keywords.
\IEEEtitleabstractindextext{%
\begin{abstract}
The abstract goes here.
\end{abstract}

% Note that keywords are not normally used for peerreview papers.
\begin{IEEEkeywords}
Computer Society, IEEEtran, journal, \LaTeX, paper, template.
\end{IEEEkeywords}}


% make the title area
\maketitle


% To allow for easy dual compilation without having to reenter the
% abstract/keywords data, the \IEEEtitleabstractindextext text will
% not be used in maketitle, but will appear (i.e., to be "transported")
% here as \IEEEdisplaynontitleabstractindextext when compsoc mode
% is not selected <OR> if conference mode is selected - because compsoc
% conference papers position the abstract like regular (non-compsoc)
% papers do!
\IEEEdisplaynontitleabstractindextext
% \IEEEdisplaynontitleabstractindextext has no effect when using
% compsoc under a non-conference mode.


% For peer review papers, you can put extra information on the cover
% page as needed:
% \ifCLASSOPTIONpeerreview
% \begin{center} \bfseries EDICS Category: 3-BBND \end{center}
% \fi
%
% For peerreview papers, this IEEEtran command inserts a page break and
% creates the second title. It will be ignored for other modes.
\IEEEpeerreviewmaketitle



\section{Introduction}
% Computer Society journal papers do something a tad strange with the very
% first section heading (almost always called "Introduction"). They place it
% ABOVE the main text! IEEEtran.cls currently does not do this for you.
% However, You can achieve this effect by making LaTeX jump through some
% hoops via something like:
%
%\ifCLASSOPTIONcompsoc
%  \noindent\raisebox{2\baselineskip}[0pt][0pt]%
%  {\parbox{\columnwidth}{\section{Introduction}\label{sec:introduction}%
%  \global\everypar=\everypar}}%
%  \vspace{-1\baselineskip}\vspace{-\parskip}\par
%\else
%  \section{Introduction}\label{sec:introduction}\par
%\fi
%
% Admittedly, this is a hack and may well be fragile, but seems to do the
% trick for me. Note the need to keep any \label that may be used right
% after \section in the above as the hack puts \section within a raised box.



% The very first letter is a 2 line initial drop letter followed
% by the rest of the first word in caps (small caps for compsoc).
% 
% form to use if the first word consists of a single letter:
% \IEEEPARstart{A}{demo} file is ....
% 
% form to use if you need the single drop letter followed by
% normal text (unknown if ever used by IEEE):
% \IEEEPARstart{A}{}demo file is ....
% 
% Some journals put the first two words in caps:
% \IEEEPARstart{T}{his demo} file is ....
% 
% Here we have the typical use of a "T" for an initial drop letter
% and "HIS" in caps to complete the first word.
\IEEEPARstart{T}{his} paragraph introduces the problem of the stock price prediction and its importance
% You must have at least 2 lines in the paragraph with the drop letter
% (should never be an issue)


A brief introduction to different techniques for price prediction then\cite{ahmadi2020comparative}.

A brief introduction to ANN models and structures for stock price prediction

A brief look at different features and information used to predict stock prices

An introduction to the current problem (different information sources suitable for different situations) and the challenges

List current work innovations and contributions

present the rest of the paper structure


\section{Literature Review}
A paragraph to start the literature review.

\subsection{Categorization of DNN models}
A brief categorization of the proposed DNN models for stock price prediction

Explain the structure of the models based on MLP network.

Explain the structure of the models based on CNN network.

Explain the structure of the models based on RNN network.

\subsubsection{Notes}
\begin{enumerate}
	\item The   existing   forecasting   methods   make   useof   both   linear   (AR,MA,ARIMA)   and   non-linear  algorithms(ARCH,GARCH,Neural Networks),but they focus on predictingthe  stock  index  movement  or  price  forecasting  for  a singlecompany  using  the  daily  closing  price.  The  proposed  methodis  a  model  independent  approach.  Here  we  are  not  fitting  thedata  to  a  specific  model,  rather  we  are  identifying  the  latentdynamics existing in the data using deep learning architectures.In this work we use three different deep learning architectures forthe price prediction of NSE listed companies and compares theirperformance.  We  are  applying  a  sliding  window  approach  forpredicting future values on a short term basis.The performanceof the models were quantified using percentage error \cite{selvin2017stock}.
	
	\item This paper proposes a novel application of deeplearning models, Paragraph Vector, and Long Short-TermMemory (LSTM), to financial time series forecasting. Investorsmake decisions according to various factors, including con-sumer price index, price-earnings ratio, and miscellaneousevents reported in newspapers. In order to assist their decisionsin a timely manner, many automatic ways to analyze thoseinformation have been proposed in the last decade. However,many of them used either numerical or textual information,but not both for a single company. In this paper, we proposean approach that converts newspaper articles into their dis-tributed representations via Paragraph Vector and models thetemporal effects of past events on opening prices about multiplecompanies with LSTM. The performance of the proposedapproach is demonstrated on real-world data of fifty companieslisted on Tokyo Stock Exchange \cite{akita2016deep}.
	
	\item We present an Artificial Neural Network (ANN) approach to predict stock market indices, particularlywith respect to the forecast of their trend movementsup or down.  Exploiting different Neural Networks archi-tectures, we provide numerical analysis of concrete financial time series.  In particular, after a brief r ́esum ́e of theexisting literature on the subject, we consider the Multi-layer Perceptron (MLP), the Convolutional Neural Net-works (CNN), and the Long Short-Term Memory (LSTM) recurrent neural networks techniques. We focus on theimportance of choosing the correct input features, along with their preprocessing, for the specific learning algo-rithm one wants to use.  Eventually, we consider the S\&P500 historical time series, predicting trend on the basisof data from the past days, and proposing a novel approach based on combination of wavelets and CNN, whichoutperforms the basic neural networks ones. We show, that neural networks are able to predict financial time seriesmovements even trained only on plain time series data and propose more ways to improve results \cite{di2016artificial}.
	
	\item We  propose  a  deep  learning  method  for  event-driven  stock  market  prediction.   First,  events  areextracted from news text, and represented as densevectors,  trained  using  a  novel  neural  tensor  net-work. Second, a deep convolutional neural networkis used to model both short-term and long-term in-fluences of events on stock price movements.  Ex-perimental results show that our model can achievenearly 6\% improvements on S\&P 500 index predic-tion  and  individual  stock  prediction,  respectively,compared to state-of-the-art baseline methods.  Inaddition,  market  simulation  results  show  that  oursystem is more capable of making profits than pre-viously reported systems trained on S\&P 500 stockhistorical data \cite{ding2015deep}.
	
	\item  This paper plans to forecast these short - term prices of stocks. 10 unique stocks recorded on New York Stock Exchange are considered for this review. The review essentially focuses on the prediction of these short - term prices leveraging the power of technical analysis. Technical Analysis guides the framework to understand the patterns from the historical prices fed into it, and attempts to probabilistically forecast the fleeting future prices of the stock under review. The paper discusses about two distinct sorts of Artificial Neural Networks, Feed Forward Neural Networks and Recurrent Neural Networks. The review uncovers that Feed Forwards Multilayer Perceptron perform superior to Long Short-Term Memory, at predicting the short - term prices of a stock \cite{khare2017short}.
	
	\item In general, stock market is very complex nonlinear dynamic system. Accordingly, accurate prediction of stock market is a very challenging task, owing to the inherent noisy environment and high volatility related to outside factors. In this paper, we focus on deep learning method to achieve high precision in stock market forecast. And a deep belief networks (DBNs), which is a kind of deep learning algorithm model, coupled with stock technical indicators (STIs) and two-dimensional principal component analysis ((2D) 2 PCA) is introduced as a novel approach to predict the closing price of stock market. A comparison experiment is also performed to evaluate this model \cite{gao2016deep}.
	
	\item Stock market is considered chaotic, complex, volatile and dynamic. Undoubtedly, its prediction is one of the most challenging tasks in time series forecasting. Moreover existing Artificial Neural Network (ANN) approaches fail to provide encouraging results. Meanwhile advances in machine learning have presented favourable results for speech recognition, image classification and language processing. Methods applied in digital signal processing can be applied to stock data as both are time series. Similarly, learning outcome of this paper can be applied to speech time series data. Deep learning for stock prediction has been introduced in this paper and its performance is evaluated on Google stock price multimedia data (chart) from NASDAQ. The objective of this paper is to demonstrate that deep learning can improve stock market forecasting accuracy. For this, (2D)2PCA + Deep Neural Network (DNN) method is compared with state of the art method 2-Directional 2-Dimensional Principal Component Analysis (2D)2PCA + Radial Basis Function Neural Network (RBFNN). It is found that the proposed method is performing better than the existing method RBFNN with an improved accuracy of 4.8\% for Hit Rate with a window size of 20. Also the results of the proposed model are compared with the Recurrent Neural Network (RNN) and it is found that the accuracy for Hit Rate is improved by 15.6\%. The correlation coefficient between the actual and predicted return for DNN is 17.1\% more than RBFNN and it is 43.4\% better than RNN \cite{singh2017stock}.
	
	\item Our study attempts to provides a comprehensive and objective assessment of both the advantages and drawbacks of deep learning algorithms for stock market analysis and prediction. Using high-frequency intraday stock returns as input data, we examine the effects of three unsupervised feature extraction methods—principal component analysis, autoencoder, and the restricted Boltzmann machine—on the network’s overall ability to predict future market behavior. Empirical results suggest that deep neural networks can extract additional information from the residuals of the autoregressive model and improve prediction performance; the same cannot be said when the autoregressive model is applied to the residuals of the network. Covariance estimation is also noticeably improved when the predictive network is applied to covariance-based market structure analysis. Our study offers practical insights and potentially useful directions for further investigation into how deep learning networks can be effectively used for stock market analysis and prediction \cite{chong2017deep}.
	
	\item In this paper, we propose a novel end-to-end model named multi-filters neural network (MFNN) specifically for feature extraction on financial time series samples and price movement prediction task. Both convolutional and recurrent neurons are integrated to build the multi-filters structure, so that the information from different feature spaces and market views can be obtained. We apply our MFNN for extreme market prediction and signal-based trading simulation tasks on Chinese stock market index CSI 300. Experimental results show that our network outperforms traditional machine learning models, statistical models, and single-structure(convolutional, recurrent, and LSTM) networks in terms of the accuracy, profitability, and stability \cite{long2019deep}.
	
	\item . This paper focus  on  architectures  such  as  Convolutional  Neural  Networks  (CNN) and Recurrent Neural Networks (RNN), which have had good  results  in  traditional  NLP  tasks.  Results  has  shown  that  CNN  can  be  better  than  RNN  on  catching  semantic  from  texts  and  RNN  is  better  on  catching  the  context  information  and  modeling  complex  temporal  characteristics  for  stock  market  forecasting.  The  proposed  method  shows  some  improvement  when compared with similar previous studies \cite{vargas2017deep}.
	
	\item Deep neural networks (DNNs) combine the advantages of deep learning (DL) and neural networks and can be used to solve nonlinear problems more satisfactorily compared to conventional machine learning algorithms. In this paper, financial product price data are treated as a one-dimensional series generated by the projection of a chaotic system composed of multiple factors into the time dimension, and the price series is reconstructed using the time series phase-space reconstruction (PSR) method. A DNN-based prediction model is designed based on the PSR method and a long- and short-term memory networks (LSTMs) for DL and used to predict stock prices. The proposed and some other prediction models are used to predict multiple stock indices for different periods. A comparison of the results shows that the proposed prediction model has higher prediction accuracy \cite{yu2020stock}.
	
	\item To date, designing a good deep learning model depends on how well the user can extract the features that represent all the characteristics of the training data. Among the various available features for training and test data, we determined that the use of event binary features can make stock price prediction models perform better. An event binary feature refers to a 0 or 1 value describing whether an indicator is satisfied (1) or not (0) for any given day and stock. We proposed and compared a stock price prediction model with three different feature combinations to verify the importance of binary features. As a result, we derived a prediction model that defeated the market (KOSPI and KODAQ (KOSPI (Korea Composite Stock Price Index) and KOSDAQ (Korean Securities Dealers Automated Quotations) is Korean stock indices)). The results suggest that deep learning is suitable for stock price prediction \cite{song2020importance}.
	
	\item This paper has performed a novel analysis of the parameter look-back period used with recurrent neural networks and also compared stock price prediction performance of three deep learning models: Vanilla RNN, LSTM, and GRU for predicting stock prices of the two most popular and strongest commercial banks listed on Nepal Stock Exchange (NEPSE). From the experiments performed, it is found that GRU is most successful in stock price prediction. In addition, the research work has suggested suitable values of the look-back period that could be used with LSTM and GRU for better stock price prediction performance \cite{saud2020analysis}.
	
	\item This paper is inspired by the recent success of using deep learning for stock market prediction. In this work, we analyze and present the characteristics of the cryptocurrency market in a high-frequency setting. In particular, we applied a deep learning approach to predict the direction of the mid-price changes on the upcoming tick. We monitored live tick-level data from 8 cryptocurrency pairs and applied both statistical and machine learning techniques to provide a live prediction. We reveal that promising results are possible for cryptocurrencies, and in particular, we achieve a consistent 78\% accuracy on the prediction of the mid-price movement on live exchange rate of Bitcoins vs US dollars \cite{fang2020ascertaining}. 
	
	\item In this paper, a comparative study isdone on two time series datasets ”Occupancy Dataset” and ”Google StockPrice Dataset”. Both these datasets are used for multivariate and univari-ate time series forecasting. Prediction was done for ’temperature’ fromOccupancy dataset and ’open’ from Google Stock Price dataset. Variousstate of the art sequence transduction, deep learning models are used.The models used are Long-Short Term Memory (LSTM), Gated Recur-rent Unit (GRU), Convolution Neural Network (CNN) and Multi-layerPerceptron (MLP). In additon, ARIMA model is used for univariate anal-ysis. LSTM and GRU models showed excellent results for multivariatetime series forecasting. In univariate analysis, all the model showed ex-cellent results , although ARIMA displayed a poor result on Google StockPrice dataset.  Mean square error is used as a metric for checking theaccuracy for the model.  LSTM and GRU models are the best modelsfor multivariate time series forecasting. Except ARIMA, all the modelsdescribed have a higher accuracy for univariate prediction \cite{tiwari2020comparative}.
	
	\item Stock price index is an essential component of financial systems and indicates the economic performance in the national level. Even if a small improvement in its forecasting performance will be highly profitable and meaningful. This manuscript input technical features together with macroeconomic indicators into an improved Stacking framework for predicting the direction of the stock price index in respect of the price prevailing some time earlier, if necessary, a month. Random forest (RF), extremely randomized trees (ERT), extreme gradient boosting (XGBoost) and light gradient boosting machine (LightGBM), which pertain to the tree-based algorithms, and recurrent neural networks (RNN), bidirectional RNN, RNN with long short-term memory (LSTM) and gated recurrent unit (GRU) layer, which pertain to the deep learning algorithms, are stacked as base classifiers in the first layer. Cross-validation method is then implemented to iteratively generate the input for the second level classifier in order to prevent overfitting. In the second layer, logistic regression, as well as its regularized version, are employed as meta-classifiers to identify the unique learning pattern of the base classifiers. Empirical results over three major U.S. stock indices indicate that our improved Stacking method outperforms state-of-the-art ensemble learning algorithms and deep learning models, achieving a higher level of accuracy, F-score and AUC value. Besides, another contribution in our research paper is the design of a Lasso (least absolute shrinkage and selection operator) based meta-classifier that is capable of automatically weighting/selecting the optimal base learners for the forecasting task. Our findings provide an integrated Stacking framework in the financial area \cite{jiang2020improved}.
	
	\item This study intends to predict the trends of price for a cryptocurrency, i.e.Ethereum based on deep learning techniques considering its trends on time seriesparticularly. This study analyses how deep learning techniques such as multi-layer perceptron (MLP) and long short-term memory (LSTM) help in predicting the pricetrends of Ethereum. These techniques have been applied based on historical datathat were computed per day, hour and minute wise. The dataset is sourced from the CoinDesk repository. The performance of the obtained models is critically assessed using statistical indicators like mean absolute error (MAE), mean squared error(MSE) and root mean squared error (RMSE) \cite{kumar2020predicting}.
	
	\item  However, very few studies have applied sequence models with robust feature engineering to predict future pricing. In this study, we investigate a framework with a set of advanced machine learning forecasting methods with a fixed set of exogenous and endogenous factors to predict daily Bitcoin prices. We study and compare different approaches using the root mean squared error (RMSE). Experimental results show that the gated recurring unit (GRU) model with recurrent dropout performs better than popular existing models. We also show that simple trading strategies, when implemented with our proposed GRU model and with proper learning, can lead to financial gain \cite{dutta2020gated}.
	
	\item In previous studies, we have used price-based input-features to measure performance changes in deep learning models. Results of this studies have revealed that the performance of stock price models would change according to varied input-features configured based on stock price. Therefore, we have concluded that more novel input-feature in deep learning model is needed to predict patterns of stock price fluctuation more precisely. In this paper, for predicting stock price fluctuation, we design deep learning model using 715 novel input-features configured on the basis of technical analyses. The performance of the prediction model was then compared to another model that employed simple price-based input-features. Also, rather than taking randomly collected set of stocks, stocks of a similar pattern of price fluctuation were filtered to identify the influence of filtering technique on the deep learning model. Finally, we compared and analyzed the performances of several models using different configuration of input-features and target-vectors \cite{song2019study}.
	
	\item In finance, the weak form of the Efficient Market Hypothesis asserts that historic stock price and volume data cannot inform predictions of future prices. In this paper we show that, to the contrary, future intra-day stock prices could be predicted effectively until 2009. We demonstrate this using two different profitable machine learning-based trading strategies. However, the effectiveness of both approaches diminish over time, and neither of them are profitable after 2009. We present our implementation and results in detail for the period 2003-2017 and propose a novel idea: the use of such flexible machine learning methods as an objective measure of relative market efficiency. We conclude with a candidate explanation, comparing our returns over time with high-frequency trading volume, and suggest concrete steps for further investigation \cite{byrd2019intraday}. 
	
	\item In  this  work  an effort  is  made  to  predict  the  price  and  price  trend  of  stocksby applying  optimal  Long  Short  Term  Memory  (O-LSTM)  deep learning and adaptive Stock Technical Indicators (STIs). We also evaluated the model for taking buy-sell decision at the end of day. To  optimize  the  deep  learning  task  we  utilized  the  concept  of Correlation-Tensorbuilt  with  appropriate  STIs.  The  tensorwith adaptive indicatorsis passed to the model for better and accurate prediction.The  results  are  analyzed  using  popular  metrics  and compared  with  two  benchmark  ML  classifiers  and  a  recent classifier based on deep learning. The mean prediction accuracy achieved using proposed model is 59.25\%, over number of stocks, which is much higher than benchmark approaches \cite{agrawal2019stock}.
	
	\item Prediction of future movement of stock prices has been a subject matter of many research work. There is a gamut of literature of technical analysis of stock prices where the objective is to identify patterns in stock price movements and derive profit from it. Improving the prediction accuracy remains the single most challenge in this area of research. We propose a hybrid approach for stock price movement prediction using machine learning, deep learning, and natural language processing. We select the NIFTY 50 index values of the National Stock Exchange (NSE) of India, and collect its daily price movement over a period of three years (2015–2017). Based on the data of 2015–2017, we build various predictive models using machine learning, and then use those models to predict the closing value of NIFTY 50 for the period January 2018 till June 2019 with a prediction horizon of one week. For predicting the price movement patterns, we use a number of classification techniques, while for predicting the actual closing price of the stock, various regression models have been used. We also build a Long and Short-Term Memory (LSTM)-based deep learning network for predicting the closing price of the stocks and compare the prediction accuracies of the machine learning models with the LSTM model. We further augment the predictive model by integrating a sentiment analysis module on Twitter data to correlate the public sentiment of stock prices with the market sentiment. This has been done using Twitter sentiment and previous week closing values to predict stock price movement for the next week. We tested our proposed scheme using a cross validation method based on Self Organizing Fuzzy Neural Networks (SOFNN) and found extremely interesting results \cite{mehtab2019robust}. 
	
	\item Mid-price movement prediction based on the limit order book data is a challenging task due to the complexity and dynamics of the limit order book. So far, there have been very limited attempts for extracting relevant features based on the limit order book data. In this paper, we address this problem by designing a new set of handcrafted features and performing an extensive experimental evaluation on both liquid and illiquid stocks. More specifically, we present an extensive set of econometric features that capture the statistical properties of the underlying securities for the task of mid-price prediction. The experimental evaluation consists of a head-to-head comparison with other handcrafted features from the literature and with features extracted from a long short-term memory autoencoder by means of a fully automated process. Moreover, we develop a new experimental protocol for online learning that treats the task above as a multi-objective optimization problem and predicts: 1) the direction of the next price movement and; 2) the number of order book events that occur until the change takes place. In order to predict the mid-price movement, features are fed into nine different deep learning models based on multi-layer perceptrons, convolutional neural networks, and long short-term memory neural networks. The performance of the proposed method is then evaluated on liquid and illiquid stocks (i.e., TotalView-ITCH US and Nordic stocks). For some stocks, results suggest that the correct choice of a feature set and a model can lead to the successful prediction of how long it takes to have a stock price movement \cite{ntakaris2019feature}.
	
	\item Energy resources have acquired a strategic significance for economic growth and social welfare of any country throughout the history. Therefore, the prediction of crude oil price fluctuation is a significant issue. In recent years, with the development of artificial intelligence, deep learning has attracted wide attention in various industrial fields. Some scientific research about using the deep learning model to fit and predict time series has been developed. In an attempt to increase the accuracy of oil market price prediction, Long Short Term Memory, a representative model of deep learning, is applied to fit crude oil prices in this paper. In the traditional application field of long short term memory, such as natural language processing, large amount of data is a consensus to improve training accuracy of long short term memory. In order to improve the prediction accuracy by extending the size of training set, transfer learning provides a heuristic data extension approach. Moreover, considering the equivalent of each historical data to train the long short term memory is difficult to reflect the changeable behaviors of crude oil markets, a very creative algorithm named data transfer with prior knowledge which provides a more availability data extension approach (three data types) is proposed. For comparing the predicting performance of initial data and data transfer deeply, the ensemble empirical mode decomposition is applied to decompose time series into several intrinsic mode functions, and these intrinsic mode functions are utilized to train the models. Further, the empirical research is performed in testing the prediction effect of West Texas Intermediate and Brent crude oil by evaluating the predicting ability of the proposed model, and the corresponding superiority is also demonstrated \cite{cen2019crude}.
	
	\item In this paper, we propose a novel stock price prediction model based on deep learning. With the success of deep learning algorithms in the field of Artificial Neural Network (ANN), we choose to solve the regression based problems (stock price prediction in our case). Stock price prediction is a challenging problem due to its random movement. This hybrid model is a combination of two well-known networks, Long Short Term Memory (LSTM) and Gated Recurrent Unit (GRU). We choose the S\&P 500 historical time series data and use significant evaluation metrics such as mean squared error, mean absolute percentage error etc., that conventional approaches have used. In experiment section, we have described the effectiveness of each of the component of our model along with its performance gain over the state-of-the-art approach. Our prediction model provides less error by considering this random nature (change) for a large scale of data \cite{hossain2018hybrid}.
	
	
	
\end{enumerate}


\subsection{Categorization of used features and information sources}
A brief categorization of the used features and information to predict stock prices. If it is possible, point to pros and cons of each information source and compare them briefly.

Each information source or category, at least one paragraph


\subsection{Current paper}
Explain the importance of using right features at each timestep according to the  market situation

Current paper importance(cite attention based models here !?)


\section{Proposed Model}
An introduction paragraph to start the section

\subsection{A description of the problem}
here a description of the problem along with the formulations goes.


\subsection{Introducing Attention mechanism}
A detailed introduction of the attention mechanism with required citation goes here.

Also The proposed attention mechanism in the current paper should be explained exactly here.

\subsection{Define the other parts of the model}
The other parts of the model should bee described here.

\subsection{Summary of the proposed model}
May be providing a summary of the proposed model with a figure of the whole architecture make sense in this subsection! (It should be double checked with Dr. Safabakhsh)

\section{Experiments}
A starting paragraph

\subsection{Datasets}
Explaining TSE and S\&P datasets (Inclusion of TSE dataset in the current paper should be double checked with Dr. Hajizadeh and Dr. Safabakhsh)


\subsection{Used indicators and features}
Explain features and indicators used in the evaluations for each of the above-mentioned datasets


\subsection{Evaluation metrics}
Introduce evaluation metrics used in the paper (If necessary)

\subsection{Evaluations and results}
Here in this subsection the detailed evaluation processes and obtained results should be reported.

\subsection{Discussion}

A brief discussion of the obtained results and conclusions made on top of them should be placed here.





% An example of a floating figure using the graphicx package.
% Note that \label must occur AFTER (or within) \caption.
% For figures, \caption should occur after the \includegraphics.
% Note that IEEEtran v1.7 and later has special internal code that
% is designed to preserve the operation of \label within \caption
% even when the captionsoff option is in effect. However, because
% of issues like this, it may be the safest practice to put all your
% \label just after \caption rather than within \caption{}.
%
% Reminder: the "draftcls" or "draftclsnofoot", not "draft", class
% option should be used if it is desired that the figures are to be
% displayed while in draft mode.
%
%\begin{figure}[!t]
%\centering
%\includegraphics[width=2.5in]{myfigure}
% where an .eps filename suffix will be assumed under latex, 
% and a .pdf suffix will be assumed for pdflatex; or what has been declared
% via \DeclareGraphicsExtensions.
%\caption{Simulation Results.}
%\label{fig_sim}
%\end{figure}

% Note that IEEE typically puts floats only at the top, even when this
% results in a large percentage of a column being occupied by floats.
% However, the Computer Society has been known to put floats at the bottom.


% An example of a double column floating figure using two subfigures.
% (The subfig.sty package must be loaded for this to work.)
% The subfigure \label commands are set within each subfloat command,
% and the \label for the overall figure must come after \caption.
% \hfil is used as a separator to get equal spacing.
% Watch out that the combined width of all the subfigures on a 
% line do not exceed the text width or a line break will occur.
%
%\begin{figure*}[!t]
%\centering
%\subfloat[Case I]{\includegraphics[width=2.5in]{box}%
%\label{fig_first_case}}
%\hfil
%\subfloat[Case II]{\includegraphics[width=2.5in]{box}%
%\label{fig_second_case}}
%\caption{Simulation results.}
%\label{fig_sim}
%\end{figure*}
%
% Note that often IEEE papers with subfigures do not employ subfigure
% captions (using the optional argument to \subfloat[]), but instead will
% reference/describe all of them (a), (b), etc., within the main caption.


% An example of a floating table. Note that, for IEEE style tables, the 
% \caption command should come BEFORE the table. Table text will default to
% \footnotesize as IEEE normally uses this smaller font for tables.
% The \label must come after \caption as always.
%
%\begin{table}[!t]
%% increase table row spacing, adjust to taste
%\renewcommand{\arraystretch}{1.3}
% if using array.sty, it might be a good idea to tweak the value of
% \extrarowheight as needed to properly center the text within the cells
%\caption{An Example of a Table}
%\label{table_example}
%\centering
%% Some packages, such as MDW tools, offer better commands for making tables
%% than the plain LaTeX2e tabular which is used here.
%\begin{tabular}{|c||c|}
%\hline
%One & Two\\
%\hline
%Three & Four\\
%\hline
%\end{tabular}
%\end{table}


% Note that IEEE does not put floats in the very first column - or typically
% anywhere on the first page for that matter. Also, in-text middle ("here")
% positioning is not used. Most IEEE journals use top floats exclusively.
% However, Computer Society journals sometimes do use bottom floats - bear
% this in mind when choosing appropriate optional arguments for the
% figure/table environments.
% Note that, LaTeX2e, unlike IEEE journals, places footnotes above bottom
% floats. This can be corrected via the \fnbelowfloat command of the
% stfloats package.



\section{Conclusion}
The conclusion goes here.





% if have a single appendix:
%\appendix[Proof of the Zonklar Equations]
% or
%\appendix  % for no appendix heading
% do not use \section anymore after \appendix, only \section*
% is possibly needed

% use appendices with more than one appendix
% then use \section to start each appendix
% you must declare a \section before using any
% \subsection or using \label (\appendices by itself
% starts a section numbered zero.)
%


%\appendices
%\section{Proof of the First Zonklar Equation}
%Appendix one text goes here.
%
%% you can choose not to have a title for an appendix
%% if you want by leaving the argument blank
%\section{}
%Appendix two text goes here.
%
%
%% use section* for acknowledgement
%\ifCLASSOPTIONcompsoc
%  % The Computer Society usually uses the plural form
%  \section*{Acknowledgments}
%\else
%  % regular IEEE prefers the singular form
%  \section*{Acknowledgment}
%\fi
%
%
%The authors would like to thank...
%

% Can use something like this to put references on a page
% by themselves when using endfloat and the captionsoff option.
\ifCLASSOPTIONcaptionsoff
  \newpage
\fi



% trigger a \newpage just before the given reference
% number - used to balance the columns on the last page
% adjust value as needed - may need to be readjusted if
% the document is modified later
%\IEEEtriggeratref{8}
% The "triggered" command can be changed if desired:
%\IEEEtriggercmd{\enlargethispage{-5in}}

% references section

% can use a bibliography generated by BibTeX as a .bbl file
% BibTeX documentation can be easily obtained at:
% http://www.ctan.org/tex-archive/biblio/bibtex/contrib/doc/
% The IEEEtran BibTeX style support page is at:
% http://www.michaelshell.org/tex/ieeetran/bibtex/
\bibliographystyle{IEEEtran}
% argument is your BibTeX string definitions and bibliography database(s)
\bibliography{ref}
%
% <OR> manually copy in the resultant .bbl file
% set second argument of \begin to the number of references
% (used to reserve space for the reference number labels box)
%\begin{thebibliography}{1}
%
%\bibitem{IEEEhowto:kopka}
%H.~Kopka and P.~W. Daly, \emph{A Guide to {\LaTeX}}, 3rd~ed.\hskip 1em plus
%  0.5em minus 0.4em\relax Harlow, England: Addison-Wesley, 1999.
%
%\end{thebibliography}

% biography section
% 
% If you have an EPS/PDF photo (graphicx package needed) extra braces are
% needed around the contents of the optional argument to biography to prevent
% the LaTeX parser from getting confused when it sees the complicated
% \includegraphics command within an optional argument. (You could create
% your own custom macro containing the \includegraphics command to make things
% simpler here.)
%\begin{IEEEbiography}[{\includegraphics[width=1in,height=1.25in,clip,keepaspectratio]{mshell}}]{Michael Shell}
% or if you just want to reserve a space for a photo:

\begin{IEEEbiography}{Michael Shell}
Biography text here.
\end{IEEEbiography}

\begin{IEEEbiography}{Ehsan Hajizadeh}
	Biography text here.
\end{IEEEbiography}

\begin{IEEEbiography}{Reza Safabakhsh}
	Biography text here.
\end{IEEEbiography}

%% if you will not have a photo at all:
%\begin{IEEEbiographynophoto}{John Doe}
%Biography text here.
%\end{IEEEbiographynophoto}

% insert where needed to balance the two columns on the last page with
% biographies
%\newpage

%\begin{IEEEbiographynophoto}{Jane Doe}
%Biography text here.
%\end{IEEEbiographynophoto}

% You can push biographies down or up by placing
% a \vfill before or after them. The appropriate
% use of \vfill depends on what kind of text is
% on the last page and whether or not the columns
% are being equalized.

%\vfill

% Can be used to pull up biographies so that the bottom of the last one
% is flush with the other column.
%\enlargethispage{-5in}



% that's all folks
\end{document}


